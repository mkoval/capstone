\documentclass[11pt,twocolumn]{article}
\usepackage{amsmath}
\usepackage[margin=1in]{geometry}
\usepackage{graphicx}
\usepackage{import}
\usepackage{subfig}

\title{Vision-Based Autonomous Ground Vehicle Navigation}
\date{}
\author{
	Michael Koval \\
	mkoval@eden.rutgers.edu
}

\begin{document}
\maketitle

\section{Introduction}
\label{sec:intro}
% 1. Overall description of IGVC
The Intelligent Ground Vehicle Competition (IGVC) is an international collegiate
robotics competition that tasks teams of undergraduate students to design,
build, and program a fully autonomous mobile ground vehicle. Specifically, the
competition consists of three distinct challenges: navigating through a winding
obstacle course, travelling between global positioning system (GPS) waypoints in
large field, and responding to messages broadcast by a Joint Architecture for
Unmanned Systems (JAUS) control system. In each of these tasks, the team has no
knowledge of the course prior to competing and navigation is further complicated
by the addition of road obstacles such as cones, barriers, switchbacks, and
potholes in the drivable regions of the course.

% 2. Alternatives to vision used by other teams
% 3. Importance of vision
Remaining in the course while autonomously navigating around these obstacles
requires that the robot has accurate knowledge of its location and objects in
its surrounding environment. Previous competitors have successfully detected
road obstacles using a combination of scanning laser range finders (LIDAR) and
stereo reconstruction. Unlike LIDAR, stereo reconstruction is capable of
providing three-dimensional data for all points in a scene, instead of only
those that are coplanar with the scanning laser rangefinder. In addition to
using stereo vision to detetct obstacles, image processing is necessary to
identify the course's painted boundaries and to locate the flags that are used
to indicate areas of safe travel.

% 4. Primary vision tasks
With the importance of computer vision clearly understood, the remainder of this
paper will discuss specific algorithms designed for use on the Navigator,
Rutgers Univeristy's entry into the 2011 Intelligent Ground Vehicle Competition.
Before discussing specific computer vision algorithms Section ~\ref{sec:robot}
describes the mechanical, sensing, and computing capabilities of the Navigator
at a high level. Making use of the onboard cameras, Section ~\ref{sec:stereo}
describes the Navigator's baseline-multiplexing stereo vision system and Section
~\ref{sec:lane} discusses monocular tracking of boundary lines. Object
recognition, such as the detection of flags and potholes, is outside the scope
of this paper and is briefly discussed as future work.

\section{Economics}
\label{sec:econ}
% ???

\section{Rutgers Navigator}
\label{sec:robot}
% 1. Brief discussion of mechanical design
% 2. Sensing capabilities
% 3. Computing capabililities and software architecture
% 4. Integration of vision with path planning (i.e. costmaps)

\section{Stereo Reconstruction}
\label{sec:stereo}
% 1. Selection of the PS3 Eye camera
% 2. Hardware and software camera synchronization
% 3. Baseline, resolution, and framerate selection
% 4. Baseline multiplexing using three cameras
% 5. Point correspondances: SSD BM on the CPU or SAD BM on the GPU?
% 6. Integration with global cost map

\section{Lane Tracking}
\label{sec:lane}
% 1. Problems with other teams' approaches
% 2. Color space transformation, ground plane assumption
% 3. Creating and using a matched pulse-width filter (!)
% 4. Non-maximal supression (!)
% 5. Model fitting?
% 6. Lane width inference?

\section{Conclusion}
\label{sec:conclusion}
% ???

\section{Future Work}
\label{sec:future}
% 1. Further analysis of GPU-accelerated stereo
% 2. Stereo and/or IMU-based ground plane detection
% 4. Use stereo depth information to mask line false positives
% 3. Basic object recognition for flags, potholes, and road obstacles
% 4. Extensive testing in a software simulation (Gazebo)

\end{document}
