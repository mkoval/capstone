\documentclass[11pt,twocolumn]{article}
\usepackage{amsmath}
\usepackage[margin=1in]{geometry}
\usepackage{graphicx}
\usepackage{import}
\usepackage{subfig}

\title{Vision-Based Autonomous Ground Vehicle Navigation}
\date{}
\author{
	Michael Koval \\
	mkoval@eden.rutgers.edu
}

\begin{document}
\maketitle

\section{Introduction}
\label{sec:intro}
% 1. Overall description of IGVC
% 2. Alternatives to vision used by other teams
% 3. Importance of vision
% 4. Primary vision tasks

\section{Economics}
\label{sec:econ}
% ???

\section{Rutgers Navigator}
% 1. Brief discussion of mechanical design
% 2. Sensing capabilities
% 3. Computing capabililities and software architecture
% 4. Integration of vision with path planning (i.e. costmaps)

\section{Stereo Reconstruction}
\label{sec:stereo}
% 1. Selection of the PS3 Eye camera
% 2. Hardware and software camera synchronization
% 3. Baseline, resolution, and framerate selection
% 4. Baseline multiplexing using three cameras
% 5. Point correspondances: SSD BM on the CPU or SAD BM on the GPU?
% 6. Integration with global cost map

\section{Lane Tracking}
\label{sec:lane}
% 1. Problems with other teams' approaches
% 2. Color space transformation, ground plane assumption
% 3. Creating and using a matched pulse-width filter (!)
% 4. Non-maximal supression (!)
% 5. Model fitting?
% 6. Lane width inference?

\section{Conclusion}
\label{sec:conclusion}
% ???

\section{Future Work}
\label{sec:future}
% 1. Further analysis of GPU-accelerated stereo
% 2. Stereo and/or IMU-based ground plane detection
% 4. Use stereo depth information to mask line false positives
% 3. Basic object recognition for flags, potholes, and road obstacles
% 4. Extensive testing in a software simulation (Gazebo)

\end{document}
